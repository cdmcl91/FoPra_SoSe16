\section{Introduction}


Self Assembly is a key mechanism for the development of bottom-up nanotechnology. Self assembly monolayers are on of the first successful implementations of such a self-assembly process. Self assembly monolayers are comprised of a single layer of molecules exhibiting large scale ordering, which have arranged themselves on a substrate and exhibit long term stability. SAMs have gained popularity, as they can be used to prepare surface without the use of UHV technology prevalent in surface science.

\subsection{Applications}

Popular applications for self assembly monolayers are the modification of surface properties, for example wetting characteristics. This can be used for a wide variety of applications, such as anti-stick layers etc. As mentioned above SAMs can be prepared with comparatively little effort. SAMs can be seen as a link between nanoscale phenomena such as self assembly and molecular interaction, and macroscopic phenomena such a wetting characteristics. Theses domains are usually hard to consider at the same time, but for SAMs both aspects will come into play as is reflected in the structure of this report.

\subsection{Structure of Report}

This report is divided into two major sections, reflecting the two experimental Procedure performed for this Lab course. The first section explores the assembly of monolayers on a substrate. The second part concerns the study of wetting characteristics of the samples prepared in the First part (or rather those prepared by our colleagues). We conclude our report with a discussion of literature on the stability of mixed monolayers.