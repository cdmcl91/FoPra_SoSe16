\section{Contact Angle Measurement}

A contact angle can be defined for a triple-phase interface, where a solid, a liquid and a gas or vapor phase meet at a single point and are at an equilibrium. This is the case for a liquid drop resting on a surface (static equilibrium) or a liquid in a pipe (dynamic equilibrium). The contact angle is the angle at the triple phase interface between the solid-gas interface and the tangent of the liquid-gas or liquid-vapor interface.

\subsection{Underlying Theory}

A drop resting on a smooth solid surface is an example of a static equilibrium triple phase interface. The interfacial energies must therefore be in equilibrium, with gravity, and with each other. Let $\gamma_{LG}$, $\gamma_{SG}$ and $\gamma_{SL}$ be the respective liquid-gas, solid-gas and solid-liquid interfacial energies. We now consider an ideal solid-liquid-gas triple-phase boundary as depicted in figure \ref{fig:triple phase}, with $\theta$ the contact angle. Without loss of generality we consider a 2 dimensional system. To obtain the equilibrium contact angle we consider a small perturbation of the displayed system. We expand the liquid solid interface by a small distance $\Delta x$ and consider the changes in energy.

\begin{equation}
\Delta E = \gamma_{SL}\, \Delta x + \gamma_{SG} \, ( - \Delta x) + \gamma_{LG} \cos{\theta} \, \Delta x
\end{equation}

at equilibrium we expect this change to be congruent to zero, hence we obtain the following energy balance.

\begin{equation}
\gamma{SG} \, \Delta x = \gamma_{SL} \, \Delta x + \gamma_{LG} \cos{\theta}\, \Delta x
\end{equation}

By eliminating the common factor $\Delta x$ we obtain a surface energy balance equation for the triple-phase boundary known as Young's equation:


\begin{equation}
\gamma{SG} = \gamma_{SL} + \gamma_{LG} \cos{\theta}
\end{equation}

Deviations from the ideal behavior associated with Young's equation can be attributed to the roughness of a non ideal surface. These can be accounted for by applying either the Wenzel or Cassie equations for homogeneous and heterogeneous surfaces respectively. In the context of this experiment we contend ourselves with the simpler model reflected in Young's equation (our samples are relatively planar).

\subsection{Measurement Procedure}

The contact angle measurements where performed optically by observing a droplet of deionized water on different samples prepared in the manner described in section II. For this purpose the the sample was placed on a three degree of freedom movable stage. 

We measured directly a contact angle at least 5 points on each sample by a contact measurement device. Dropping a liquid on the surface of the sample and we plotted the liquid drop profiles and extrapolated the contact angles. Analyzing experiment data we calculated the mean values and the errors. We show the results in the next section.

\subsection{Results and Analysis}